\documentclass{article}
\usepackage[top=1in,bottom=1in]{geometry}
\usepackage{hyperref}
\usepackage{amsmath}
\usepackage{amssymb}
\usepackage{graphicx}
\def\upint{\mathchoice%
    {\mkern13mu\overline{\vphantom{\intop}\mkern7mu}\mkern-20mu}%
    {\mkern7mu\overline{\vphantom{\intop}\mkern7mu}\mkern-14mu}%
    {\mkern7mu\overline{\vphantom{\intop}\mkern7mu}\mkern-14mu}%
    {\mkern7mu\overline{\vphantom{\intop}\mkern7mu}\mkern-14mu}%
  \int}
\def\lowint{\mkern3mu\underline{\vphantom{\intop}\mkern7mu}\mkern-10mu\int}
\graphicspath{ {./assets/} }
\usepackage[none]{hyphenat}
\setlength{\parindent}{0pt}
\date{}
\title{MATH410 Advanced Calculus I}
\begin{document} 
  \author{Michael Li}
  \title{MATH410 Advanced Calculus I}
  \maketitle
  \tableofcontents
  \newpage
  \section{Chapter 6 Integrals}
  Basic idea of what an integral represents:

  For an integratable $f \colon [a,b] \rightarrow \mathbb{R}$ with $f(x) \geq 0$ for all $x \in [a, b]$,
  
  $\int_a^bf$ is the area under $f$ and above the interval $[a, b]$
  \subsection{Darboux Sums}
  For reals $a < b$, $n \in \mathbb{N}$, and $a=x_0, \ldots x_{n} = b$,

  $P = \{x_0, \ldots, x_n\}$ is a \textbf{partition} of the interval $[a, b]$.

  For an index $i \geq 0$, $x_1$ is called a \textbf{partition point} of $P$

  For $i \geq 1$, $[x_{i-1}, x_i]$ is a \textbf{partition interval} of $P$.
  \newline \newline
  Suppose $f \colon [a, b] \rightarrow \mathbb{R}$ is bounded and $P = \{x_0, \ldots, x_n\}$ is a partition of $[a, b]$, then for $i \geq 1$:

  \[m_i = \inf\{f(x) \colon x \in [x_{i-1}, x_i]\}\]
  \[M_i = \sup\{f(x) \colon x \in [x_{i-1}, x_i]\}\]
  We use $m_i$ and $M_i$ to define the \textbf{Lower and Upper Darboux Sums} for $f$ based on partition $P$
  \[L(f, P) = \sum_{i = 1}^{n}m_i(x_i - x_{i-1})\]
  \[U(f, P) = \sum_{i = 1}^{n}M_i(x_i - x_{i-1})\]
  Since $m_i \leq M_i$ for each $i \geq 1$, we have
  \[L(f, P) \leq U(f, P) \text{ for any partition } P \text{ of } [a,b]\]
  Also useful to note that for any partition of $[a, b]$:
  \[b - a = \sum_{i = 1}^{n}(x_i - x_{i-1})\]
  \textbf{Lemma 6.1}: if $f \colon [a, b] \rightarrow \mathbb{R}$ is bounded and for $m, M$: $m \leq f(x) \leq M \text{ for all } x \in [a,b]$ then 
  \[m(b-a) \leq L(f, P) \text{ and } U(f, P) \leq M(b-a)\]
  \textbf{Proof}: $m$ is a lower bound for all $m_i$ as defined above so we have:
  \[m(b-a) = \sum_{i = 1}^{n}m(x_i-x_{i-1}) \leq \sum_{i=1}^{n}{m_i}(x_i-x_{i-1}) = L(f, P)\]
  Similar proof is applied for $U(f, P) \leq M(b-a)$ \newline \newline
  Partition $P^*$ is the \textbf{refinement} of partition $P$ if each partition pt of $P$ is a partition pt of $P^*$

  Partition pts of $P^*$ that belong to the partition interval $[x_{i-1}, x_i]$ define a partition interval $P_i$. Observe:
  \[\sum_{i=1}^{n}L(f, P_i) = L(f, P^*) \text{ and } \sum_{i = 1}^{n}U(f, P_i) = U(f, P^*)\]
  \textbf{Lemma 6.2 Refinement Lemma}: Suppose $f \colon [a, b] \rightarrow \mathbb{R}$ is bounded and that $P$ is a partition of $[a, b]$. If $P^*$ is a refinement of $P$ then:
  \[L(f, P) \leq L(f, P^*) \text{ and } U(f, P^*) \leq U(f, P)\]
  \textbf{Proof}: Let $P = \{x_0, \ldots, x_n\}$, for index $i \geq 1$, define $m_i$ the same as above, and let $P_i$ be the partition of $[x_{i-1}, x_i]$ induced by $P^*$. Applying Lemma 6.1 to $f \colon [x_{i-1}, x_i] \rightarrow \mathbb{R}$ gives:
  \[m_i(x_i - x_{i-1}) \leq L(f, P_i)\]
  Taking the sum of all $n$ sub-partitions gives
  \[L(f, P) \leq \sum_{i = 1}^{n}L(f, P_i) = L(f, P^*)\]
  Similar argument can be done to show $U(f, P^*) \leq U(f,P)$ \newline \newline
  Given partitions $P_1, P_2$ of $[a,b]$, a \textbf{common refinement} $P^*$ can be formed by taking the union of the partition pts of $P_1, P_2$

  \textbf{Lemma 6.3}: Suppose $f \colon [a, b] \rightarrow \mathbb{R}$ is bounded and that $P_1, P_2$ are partitions of $[a, b]$ then:
  \[L(f, P_1) \leq U(f, P_2)\]
  \textbf{Proof}: Let $P^*$ be the common refinement of $P_1, P_2$. By the Refinement Lemma, we have:
  \[L(f, P_1) \leq L(f, P^*) \leq U(f, P^*) \leq U(f, P_2)\]
  \subsection{Lower and Upper Integrals}
  Suppose $f \colon [a, b] \rightarrow \mathbb{R}$ is bounded. We can define the lower and upper integrals on $[a,b]$ as:
  \[\lowint_{a}^{b}f = \sup\{L(f, P) \colon P \text{ is a partition of } [a,b]\}\]
  \[\upint_{a}^{b}f = \inf\{U(f, P) \colon P \text{ is a partition of } [a,b]\}\]
  \textbf{Lemma 6.4}: for a bounded $f \colon [a, b] \rightarrow \mathbb{R}$
  \[\lowint_{a}^{b}f \leq \upint_{a}^{b}f\]
  \textbf{Proof}: Let $P$ be a partition of $[a, b]$. By Lemma 6.3, $U(f, P)$ is an upperbound for all Lower Darboux Sums of $f$. Thus:
  \[\lowint_{a}^{b}f \leq U(f, P)\]
  This inequality implies that $\lowint_{a}^{b}$ is a lower bound for all Upper Darboux Sums of $f$. Thus, by definition of infimum:
  \[\lowint_{a}^{b}f \leq \upint_{a}^{b}f\]
  \subsection{Archimedes-Riemann Theorem}
  Suppose $f \colon [a,b] \rightarrow \mathbb{R}$ is bounded. Then $f$ is \textbf{integrable} on $[a,b]$ if
  \[\lowint_{a}^{b}f = \upint_{a}^{b}f\]
  When this condition is met, the integral of $f$ over $[a,b]$ is defined as
  \[\lowint_{a}^{b}f = \int_{a}^{b}f = \upint_{a}^{b}f\]
  \textbf{Lemma 6.7}: For a bounded $f \colon [a, b] \rightarrow \mathbb{R}$ and partition $P$ of $[a,b]$,
  \[L(f, P) \leq \lowint_{a}^{b}f \leq \upint_{a}^{b}f  \leq U(f, P)\]
  This creates 3 useful inequalities
  \[ 0 \leq \upint_{a}^{b}f  - \lowint_{a}^{b}f \leq U(f, P) - L(f, P)\]
  \[ 0 \leq U(f, P) - \upint_{a}^{b}f \leq U(f, P) - L(f, P)\]
  \[ 0 \leq \lowint_{a}^{b}f  - L(f, P) \leq U(f, P) - L(f, P)\]
  \textbf{Proof}: by definition of lower and upper integrals
  \[L(f, P) \leq \lowint_{a}^{b} f \text{ and } \upint_{a}^{b} \leq U(f, P)\]
  Using Lemma 6.4 we get:
  \[L(f, P) \leq \lowint_{a}^{b}f \leq \upint_{a}^{b}f  \leq U(f, P)\]
  \textbf{Theorem 6.8 Archimedes-Riemann Theorem}: Let $f \colon [a,b] \rightarrow \mathbb{R}$ be bounded. Then $f$ is integrable on $[a, b]$ if and only if there is a sequence $\{P_n\}$ of partitions of $[a,b]$ such that
  \[\lim_{n \rightarrow \infty}[U(f, P_n) - L(f, P_n)] = 0\]
  Moreover, for any such sequence of partitions,
  \[\lim_{n \rightarrow \infty}L(f, P_n) = \int_a^bf \text{ and} \lim_{n \rightarrow \infty}U(f, P_n) = \int_a^bf\]
  \textbf{Proof} Forward: suppose that such a sequence of partitions exists satisfying the equation. Using Lemma 6.7, for an index $n$, $P_n$ satisfies the inequality
  \[ 0 \leq \upint_{a}^{b}f  - \lowint_{a}^{b}f \leq \lim_{n \rightarrow \infty}[U(f, P_n) - L(f, P_n)] = 0\]
  Thus 
  \[\lowint_{a}^{b}f = \upint_{a}^{b}f \text{ as desired and } f \text{ is integrable over } [a,b]\]
  \textbf{Proof} Backwards: fix a natural number $n$. By definition of lower integral and least upper bound, 
  \[\left(\lowint_{a}^{b}f\right) - 1/n \text{ is not an upper bound for the Lower Darboux Sums of } f\]
  Thus for some partition $P'$ of $[a,b]$
  \[\lowint_{a}^{b}f - 1/n  < L(f, P')\]
  Similar for upper integral and some partition $P''$ of $[a, b]$
  \[U(f, P'') < \left(\upint_{a}^{b}f\right) + 1/n\]
  By the Refinement Lemma, the 2 inequalities above hold for a common refinement, $P_n$, of $P', P''$. Thus,
  \[ 0 \leq U(f, P_n) - L(f, P_n) < \left[\left(\upint_{a}^{b}f\right) + 1/n\right] - \left[\left(\lowint_{a}^{b}f\right) - 1/n\right] = 2/n\]
  Thus,
  \[\lim_{n \rightarrow \infty}[U(f, P_n) - L(f, P_n)] = 0\]
  The sequence of partitions $\{P_n\}$ that satisfies the Archimedes-Riemann Theorem is called the \textbf{Archimedean sequence of partitions}, satisfying:
  \[\lim_{n \rightarrow \infty}[U(f, P_n) - L(f,P_n)] = 0\]
  For a $n \in \mathbb{N}$, the partition $P = \{x_0, \ldots, x_n\}$ of $[a, b]$ is called the \textbf{regular partition} of $[a,b]$ into $n$ partition intervals (of length $(b-a)/n)$ if:
  \[x_i = a + i\frac{b - a}{n} \text { for } 0 \leq i \leq n\]
  For partition $P$ of $[a,b]$, the \textbf{gap} of $P$, denoted $\text{gap }P$, is the length of the largest partition interval:
  \[\text{gap }P = \max_{i \leq i \leq n}[x_i - x_{i-1}]\]
  \textbf{Example 6.9}: A montonically increasing $f \colon [a, b] \rightarrow \mathbb{R}$ is integrable. Let $P = \{x_0, \ldots, x_n\}$ be a partition of $[a, b]$. Since $f$ is monotonically decreasing, we can define for any index $i \geq 1$:
  \[m_i = \inf\{f(x) \colon x \in [x_{i-1}, x_i] = f(x_{i-1})\]
  \[M_i = \sup\{f(x) \colon x \in [x_{i-1}, x_i] = f(x_{i})\]
    Divide $P$ into regular partitions and for $n \in \mathbb{N}$, take $P_n$. Then
    \[U(f,P_n) - L(f, P_n) = \sum_{i = 1}^{n}(M_i - m_i)(x_i - x_{i-1}) = \sum_{i = 1}^{n}(M_i - m_i)\frac{b-a}{n} =\frac{b-a}{n}(f(b) - f(a))\]
    Thus
    \[\lim_{n \rightarrow \infty}[U(f, P_n) - L(f, P_n)] = \lim_{n \rightarrow \infty}\frac{(f(b) - f(a))(b-a)}{n} = 0\]
    \textbf{Example 6.11}: Let $f(x) = x^2$ for $x \in [0, 1]$. Show that
    \[\int_{0}^{1}x^2dx = 1/3\]
    Since $f(x)$ is monotonically increasing, it is integrable on $[0, 1]$ (shown in Example 6.9)

    Define a regular partition $P_n$ of $[0,1]$. Since $\{P_n\}$ is an Archimedean sequence of partitions for $f$ on $[0,1]$ (Example 6.9), by Archimedes-Rieman Theorem:
    \[\int_{0}^{1}x^2dx = \lim_{n \rightarrow \infty}U(f, P_n)\]
    For index $i \geq 1$, we have
    \[M_i = \sup\{f(x) \colon x \in [x_{i-1}, x_i]\} = f(x_i) = i^2/n^2\]
    \[x_i - x_{i-1} = 1/n\]
    Thus,
    \[M_i(x - x_{i-1}) = i^2/n^3\]
    Using the sum of squares we get
    \[U(f, P_n) = \sum_{i = 1}^{n}M_i(x_i - x_{i-1}) = (1/n^3)\left(\sum_{i=1}^{n}i^2\right) = \frac{n(n+1)(2n+1)}{6n^3}\]
    Therefore,
    \[\int_{0}^1x^2dx = \lim_{n \rightarrow \infty}U(f, P_n) = \lim_{n \rightarrow \infty}\frac{n(n+1)(2n+1)}{6n^3} = 1/3\]
    \subsection{Additivity, Monotonicity, Linearity}
    \textbf{Theorem 6.12 Additivity over Intervals}: Let $f \colon [a, b] \rightarrow \mathbb{R}$ be integrable over $[a, b]$ and let $c \in (a, b)$. Then $f$ is integrable over $[a,c]$ and $[c, b]$, and
    \[\int_a^b f = \int_a^c f+ \int_c^b f\]
    \textbf{Proof}: since $f$ is integrable on $[a, b]$, by the Archimedes-Riemann Theorem, there is an Archimedean sequence of partitions $\{P_n\}$ for $f$ on $[a, b]$ such that
    \[\lim_{n \rightarrow \infty}[U(f, P_n) - L(f, P_n)] = 0\]
    and
    \[\lim_{n \rightarrow \infty}U(f, P_n) = \int_a^b f\]
    Since $c$ belongs to each partition $P_n$, let $P'_n$ be the partition induced on $[a, c]$ and $P''_n$ be the partition induced on $[c,b]$. Then from the definition of Darboux Sums
    \[U(f, P_n) = U(f, P'_n) + U(f, P''_n)\]
    \[L(f, P_n) = L(f, P'_n) + L(f, P''_n)\]
    Thus
    \[U(f, P_n) - L(f, P_n) = [U(f, P'_n) - L(f, P'_n)] + [U(f, P''_n) - L(f, P''_n)]\]
    Since $P_n$ is an Archiemedean sequence of partitions, the limits of the two terms in brackets is $0$. Thus, by Archimedes-Riemann Theorem, $f$ is integrable on $[a, c]$ and on $[c, b]$, and
    \[\lim_{n \rightarrow} U(f, P'_n) = \int_a^c f \text{ and } \lim_{n \rightarrow \infty}U(f, P''_n) = \int_c^b f\]
    Thus, combining the limits we have
    \[\int_a^b f = \lim_{n \rightarrow \infty}U(f, P_n) = \int_a^c f + \int_c^b f\]
    \textbf{Theorem 6.13 Monotonicity of Integrals}: let $f \colon [a, b] \rightarrow \mathbb{R}$ and $g \colon [a, b] \rightarrow \mathbb{R}$ be integrable and $f(x) \leq g(x)$ for all $x \in [a,b]$. Then

    \[\int_a^b f \leq \int_a^b g\]
    \textbf{Proof}: By the Archimedes-Riemann Theorem and Refinement Theorem, there exists a sequence of partitions $\{P_n\}$ such that $\lim_{n \rightarrow \infty}U(f, P_n) = \int_a^b f \text{ and  } = \int_a^b g$

    Since $f(x) \leq g(x)$ by hypothesis, by the definition of Upper Darboux Sums,
    \[U(f, P_n) \leq U(g, P_n) \text{ for each index } n\]
    Thus
    \[\int_a^b f = \lim_{n \rightarrow \infty}U(f, P_n) \leq \lim_{n \rightarrow \infty}U(g, P_n) = \int_a^b g\]
    \textbf{Lemma 6.14}: let $f \colon [a, b] \rightarrow \mathbb{R}$ and $g \colon [a, b] \rightarrow \mathbb{R}$ by bounded functions and let $P$ be a partition of $[a, b]$. Then
    \[L(f, P) + L(g, P) \leq L(f + g, P) \text{ and } U(f + g, P) \leq U(f, P) + U(g, P)\]
    \[U(\alpha f, P) = \alpha U(f, P) \text{ and } L(\alpha f, P) = \alpha L(f, P) \text{ if } \alpha \geq 0\]
    \[U(\alpha f, P) = \alpha L(f, P) \text{ and } L(\alpha f, P) = \alpha U(f, P) \text{ if } \alpha \leq 0\]
    \textbf{Proof}: Choose an interval $I_i$ of partition $P$ and for a bounded function $h \colon [a,b] \rightarrow \mathbb{R}$ define
    \[M_i(h) = \sup\{h(x) \colon x \in I_i\} \text{ and } m_i(h) = \inf\{h(x) \colon x \in I_i\}\]
    Thus for any $x \in I_i$,
    \[f(x) + g(x) \leq M_i(f) + M_i(g)\]
    By definition of supremum, we have
    \[M_i(f + g) \leq M_i(f) + M_i(g)\]
    Finally, multiplying this inequality by the length of $I_i$ and summing over all intervals of $P$ yields the first part of the lemma

    To prove the second part, notice that
    \[M_i(\alpha f) = \alpha M_i(f) \text{ and } m_i(\alpha f) = \alpha m_i(f) \text{ if } \alpha \geq 0\]
    \[M_i(\alpha f) = \alpha m_i(f) \text{ and } m_i(\alpha f) = \alpha M_i(f) \text{ if } \alpha \leq 0\]
    Then summing over all intervals, we get the second part of the lemma\bigskip

    \textbf{Theorem 6.15 Linearity of Integrals}: Let $f\colon [a, b] \rightarrow \mathbb{R}$ and $g \colon [a,b] \rightarrow \mathbb{R}$ by integrable. Then for any numbers $\alpha, \beta$ the function $\alpha f + \beta g \colon [a, b[ \rightarrow \mathbb{R}$ is integrable and
    \[\int_a^b[\alpha f + \beta g] = \alpha \int_a^b f + \beta \int_a^b g\]
    \textbf{Proof}: By Archimedes-Riemann Theorem and the Refinement Lemma, there is a sequence $\{P_n\}$ of partitions of $[a,b]$ that is both an Archimedean sequence of partitions of $f$ and for $g$ on $[a, b]$

    \textbf{Case 1}: $\beta = 0$. For an index $n$, we have
    \[U(\alpha f, P_n) - L(\alpha f, P_n) = |\alpha|U(f, P_n) - L(f, P_n)]\]
    Since $\{P_n\}$ is an Archimedean sequence of partitions for $f$ on $[a, b]$ it is also an Archmidean sequence of partitions for $\alpha f$ on $[a, b]$. By Lemma 6.14
    \[U(\alpha f, P_n) = \begin{cases} \alpha U(f, P_n) & \alpha \geq 0 \\ \alpha L(f, P_n) & \alpha \leq 0 \end{cases}\]
    However, by Archimedean-Riemann Theorem, for $\alpha \geq 0$
    \[\int_a^b \alpha f = \lim_{n \rightarrow \infty}U(\alpha f, P_n) = \alpha \lim_{n \rightarrow \infty} U(f< P_n) = \alpha \int_a^b f\]
    If $\alpha \leq 0$
    \[\int_a^b \alpha f = \lim_{n \rightarrow \infty}U(\alpha f, P_n) = \alpha \lim_{n \rightarrow \infty} L(f< P_n) = \alpha \int_a^b f\]
    \textbf{Case 2}: $\alpha = \beta = 1$. For an index $n$, we have
    \[L(f, P_n) + L(g, P_n) \leq L(f + g, P_n) \leq U(f + g, P_n) \leq U(f, P_n) + U(g, P_n)\]
    From Archimedes-Riemann Theorem, the sequence of partitions satisfy
    \[\lim_{n \rightarrow \infty} L(f + g, P_n) = \lim_{n \rightarrow \infty} U(f + g, P_n) = \int_a^b f + \int_a^b g\]
    Thus we have
    \[\int_a^b [f + g] = \int_a^b f + \int_a^b g\]
    \textbf{General case}: follow by some combination of case 1 and case 2 \bigskip

    \textbf{Corollary 6.16}: Let $f \colon [a, b] \rightarrow \mathbb{R}$ and $|f| \colon [a,b] \rightarrow \mathbb{R}$ be integrable. Then
      \[\left|\int_a^b f (x) dx \right| \leq \int_a^b |f(x)| dx\]
      \textbf{Proof}: For all $x \in [a, b]$,
      \[-|f(x)| \leq f(x) \leq |f(x)|\]
      Using the monotonicity and linearity of integration, we have
      \[-\int_a^b |f(x)| dx \leq \int_a^b f(x)dx \leq \int_a^b |f(x) dx\]
      Thus by definition of absolute value, the corollary holds.
    \subsection{Continuity and Integrability}
    \textbf{Lemma 6.17}: Let $f \colon [a,b] \rightarrow \mathbb{R}$ be continuous and let $P$ be a partition of $[a,b]$. Then there is a partition interval of $P$ that contains the points $u,v$ where
    \[0 \leq U(f, P) - L(f, P) \leq [f(v) - f(u)][b-a]\]
    \textbf{Proof}: Let $P = \{x_0, \ldots, x_n\}$. For index $i \geq 1$, since $f$ is continuous on the closed bounded partition interval $[x_{i-1}, x_i]$, by the Extreme Value Theorem, this partition interval has a max value and a min value. Thus there are points $u_i$ and $v_i$ in $[x_{i-1}, x_i]$ such that

    \[f(u_i) = m_i = \inf\{f(x) \colon x \in [x_{i-1}, x_i]\}\]
    \[f(v_i) = M_i = \sup\{f(x) \colon x \in [x_{i-1}, x_i]\}\]
    Choosing an index $i_0$ such that
    \[M_{i_0} - m_{i_0} = \max_{1 \leq i \leq n}[M_i - m_i]\]
    and defining $u = u_{i_0}$ and $v = v_{i_0}$ we have
    \[M_i - m_i \leq M_{i_0} - m_{i_0} = f(v) - f(u) \text{ for } i \leq i_0 \leq n\]
    Thus we have
    \[U(f, P) - L(f, P) = \sum_{i = 1}^n [M_i - m_i][x_i - x_{i-1}] \leq \sum_{i = 1}^n[f(v) - f(u)][x_i - x{i-1}] = [f(v) - f(u)][b-a]\]
    \textbf{Theorem 6.18}: A continuous function on a closed bounded interval is integrable

    \textbf{Proof}: Let $\{P_n\}$ be any sequence of partitions of $[a,b]$ such that $\lim_{n \rightarrow \infty} \text{gap } P_n = 0$

    Since $u_n$ and $v_n$ belong to the common partition interval $P_n$

    \[|v_n - u_n| \leq \text{ gap } P_n\]
    Furthermore, since the function is continuous we know that
    \[\lim_{n \rightarrow \infty}[u_n - v_n] = 0\]
    Since $f$ is continuous on a closed bounded interval, by Theorem 3.17, it is uniformly continuous and thus
    \[\lim_{n \rightarrow \infty}[f(u_n) - f(v_n)] = 0\]
    Using Lemma 6.17, we have
    \[0 \leq \lim_{n \rightarrow \infty}[U(f, P_n) - L(f, P_n)] \leq \lim_{n \rightarrow \infty}[f(v_n) - f(u_n)][b-a] = 0\]
    Thus the sequence $\{P_n\}$ is an Archimedean sequence of partitions for $f$ on $[a,b]$. Thus by the Archimedes-Riemann Theorem, $f$ is integrable on $[a,b]$ \bigskip

    \textbf{Theorem 6.19}: If $f \colon [a, b] \rightarrow \mathbb{R}$ is bounded an on the closed interval $[a,b]$ and is continuous on the open interval $(a,b0$ then $f$ is integrable on $[a,b]$ and the value of $\int_a^b f$ does not depend on the values of $f$ at the endpoints of the interval.
\end{document}
